%%%%%%%%%%%%%%%%%%%%%%%%%%%%%%%%%%%%%%%%%%%%%%%%%%%%%%%%%%%%%%%%%%%%%%
% Problem statement
\begin{statement}[
  timelimit=1 s,
  memorylimit=512 MiB,
]{A: Alergični Aron}

Nekadašnji najbolji prijatelj gospodina Malnara, Aron, napustio je domovinu te
je bolju budućnost potražio na jedom pustom, dalekom otoku. Zasigurno se pitate
zašto je odabrao baš takvu lokaciju, a ne neki velegrad gdje će, pod okriljem
neke glomazne koorporacije, ostvariti veliku karijeru. Naime, bolju budućnost
traži tamo gdje nema malenih paradajzića (tzv.\ \textit{cherry rajčica}) niti
ambrozije na koju je izuzetno alergičan. Kako bi mu napakostio, gospodin Malnar
u svom je uredu uzgojio biljku ambrozije.

Iako ambrozija nije stablo, zanimljivo je da se Malnarova biljka može prikazati
kao stablo s $n$ čvorova koji su povezani pomoću $(n-1)$ grana. Prisjetimo se,
stablo je neusmjereni, povezani graf u kojem između svaka dva čvora postoji
jedinstven put. Poznato je da su
alergeni koncentrirani upravo na granama, ali nisu sve grane jednako potentne.
Gospodin Malnar zna da grana koja povezuje čvorove $u_i$ i $v_i$ ima
\textit{alergičnost} $w_i$. Shodno tome, iz biljke će izrezati povezani podskup
grana najveće alergičnosti. Alergičnost podskupa definira se kao umnožak broja
grana unutar njega sa alergičnošću najnealergičnije grane unutar tog podskupa,
tj. grane s minimalnom vrijednosti $w_i$. Gospodin Malnar je nepogrešiv i odmah
je pronašao podskup s najvećom alergičnosti.

Možete li i vi odrediti taj podskup?

%%%%%%%%%%%%%%%%%%%%%%%%%%%%%%%%%%%%%%%%%%%%%%%%%%%%%%%%%%%%%%%%%%%%%%
% Input
\subsection*{Ulazni podaci}
U prvom je retku prirodni broj $n$ $(1 \le n \le 10^5)$.

U sljedećih se $n-1$ redaka nalaze brojevi $u_i$, $v_i$ i $w_i$
$(1 \le u_i$, $v_i \le n$, $u_i \neq v_i$, $1 \le w_i \le 10^9)$ koji
predstavljaju grane kako je opisano u tekstu zadatka.

%%%%%%%%%%%%%%%%%%%%%%%%%%%%%%%%%%%%%%%%%%%%%%%%%%%%%%%%%%%%%%%%%%%%%%
% Output
\subsection*{Izlazni podaci}
U jedini redak ispišite alergičnost najalergičnijeg povezanog podskupa
grana ambrozije.

%%%%%%%%%%%%%%%%%%%%%%%%%%%%%%%%%%%%%%%%%%%%%%%%%%%%%%%%%%%%%%%%%%%%%%
% Examples
\subsection*{Probni primjeri}
  TODO
%\begin{tabularx}{\textwidth}{XX'XX}
    %\textbf{ulaz}
  %\linespread{1}{\verbatiminput{test/reality.dummy.in.1}} &
  %\textbf{izlaz}
  %\linespread{1}{\verbatiminput{test/reality.dummy.out.1}} &
  %\textbf{ulaz}
  %\linespread{1}{\verbatiminput{test/reality.dummy.in.2}} &
  %\textbf{izlaz}
  %\linespread{1}{\verbatiminput{test/reality.dummy.out.2}}
%\end{tabularx}

%%%%%%%%%%%%%%%%%%%%%%%%%%%%%%%%%%%%%%%%%%%%%%%%%%%%%%%%%%%%%%%%%%%%%%
% We're done
\end{statement}

%%% Local Variables:
%%% mode: latex
%%% mode: flyspell
%%% ispell-local-dictionary: "croatian"
%%% TeX-master: "../studentsko2018.tex"
%%% End:

%%%%%%%%%%%%%%%%%%%%%%%%%%%%%%%%%%%%%%%%%%%%%%%%%%%%%%%%%%%%%%%%%%%%%%
% Problem statement
\begin{statement}[
  timelimit=2 s,
  memorylimit=512 MiB,
]{B: Bliski Brojevi}

Opće je poznato da je gospodin Malnar član \textit{Mense} -- međunarodne
organizacije u koju se moguće učlaniti isključivo rezultatom na standardiziranom
IQ testu koji je bolji od $98\%$ ukupne populacije. Možda je manje poznato da je
gospodin Malnar također član \textit{Menze} -- međunarodne organizacije u koju
se moguće učlaniti isključivo ako je vaš godišnji unos standardnih porcija u
restoranima viši od $98\%$ populacije. Jednom se prilikom i sam gospodin Malnar
zabunio, pa je pri ulasku u Menzu predočio člansku iskaznicu Mense. Vijest se
brzo širila hodnicima Menze pa je dio znatiželjnih članova odlučio testirati
kognitivne sposobnosti gospodina Malnara nakon idućeg grupnog objeda.

Znatiželjnici su skupili prazne tanjure i od njih napravili $n$ hrpa tako
da se prva hrpa sastojala od jednog tanjura, druga od dva, i tako sve do
$n$-te hrpe koja se sastojala od $n$ tanjura. Potom su te hrpe ispromiješali, a
gospodin Malnar je trebao odgovarati na $q$ brzopoteznih pitanja. Svako pitanje
je bilo istog oblika, a glasilo je: ,,Kolika je najmanja razlika u broju tanjura
nekih dvaju hrpa koje se nalaze između $l$-te i $r$-te hrpe?''. Formalno, neka
je broj tanjura $i$-te hrpe označen s $p_i$, tada gospodin Malnar treba odrediti:
$$\min_{l \le i < j \le r}{|p_i - p_j|}$$

Zanimljivo je da je gospodin Malnar na sva pitanja odgovorio unutar jedne
sekunde te da pritom nije potrošio više od $512$ MiB memorije. Vi to
zasigurno ne možete, ali možda biste mogli napisati takav program.

%%%%%%%%%%%%%%%%%%%%%%%%%%%%%%%%%%%%%%%%%%%%%%%%%%%%%%%%%%%%%%%%%%%%%%
% Input
\subsection*{Ulazni podaci}
U prvom su retku prirodni brojevi $n$ i $q$ $(1 \le n, q \le 3 \cdot 10^4)$ iz
teksta zadatka.\\
U drugom se retku nalazi permutacija $p$ duljine $n$ iz teksta zadatka.\\
U sljedećih se $q$ redaka nalaze po dva prirodna broja $l$ i $r$
$(1 \le l < r \le n)$ iz teksta zadatka.

%%%%%%%%%%%%%%%%%%%%%%%%%%%%%%%%%%%%%%%%%%%%%%%%%%%%%%%%%%%%%%%%%%%%%%
% Output
\subsection*{Izlazni podaci}
U $i$-tom retku ispišite odgovor na $i$-ti upit iz ulaza.

%%%%%%%%%%%%%%%%%%%%%%%%%%%%%%%%%%%%%%%%%%%%%%%%%%%%%%%%%%%%%%%%%%%%%%
% Examples
\subsection*{Probni primjeri}
\begin{tabularx}{\textwidth}{X'X}
  \textbf{ulaz}
  \linespread{1}{\verbatiminput{test/B.dummy.in.1}} &
  \textbf{izlaz}
  \linespread{1}{\verbatiminput{test/B.dummy.out.1}}
\end{tabularx}

%%%%%%%%%%%%%%%%%%%%%%%%%%%%%%%%%%%%%%%%%%%%%%%%%%%%%%%%%%%%%%%%%%%%%%
% We're done
\end{statement}

%%% Local Variables:
%%% mode: latex
%%% mode: flyspell
%%% ispell-local-dictionary: "croatian"
%%% TeX-master: "../studentsko2018.tex"
%%% End:

%%%%%%%%%%%%%%%%%%%%%%%%%%%%%%%%%%%%%%%%%%%%%%%%%%%%%%%%%%%%%%%%%%%%%%
% Problem statement
\begin{statement}[
  timelimit=1 s,
  memorylimit=512 MiB,
]{C: Crni Ceh}

Na jednom programerskom natjecanju sudjeluje $n$ natjecatelja. Prije natjecanja
svaki je natjecatelj od gospodina Malnara dobio majicu. Neki su natjecatelji
dobili žute, a neki crne majice, čime je nastalo rivalstvo između crnog i žutog
tima.

Na početku natjecanja svi natjecatelji imaju $0$ bodova te je za svakoga poznata
boja njegove majice. Tijekom natjecanja dogodilo se $q$ promjena u rezultatima.
U $i$-toj promjeni je natjecatelj $x_i$ upravo dobio još $d_i$ bodova.

Svaki natjecatelj u žutoj majici računa svoju kaznu (tzv.\ \textit{crni ceh})
kao broj natjecatelja u crnoj majici koji u tom trenutku imaju strogo više
bodova od njega. Izračunajte i ispišite koliki je zbroj kazni natjecatelja u
žutim majicama nakon svake promjene u bodovima.

%%%%%%%%%%%%%%%%%%%%%%%%%%%%%%%%%%%%%%%%%%%%%%%%%%%%%%%%%%%%%%%%%%%%%%
% Input
\subsection*{Ulazni podaci}
U prvom su retku prirodni brojevi $n$ $(1 \le n \le 10^5) $ i
$q$ $(1 \le q \le 3 \cdot 10^5)$ iz teksta zadatka.

U sljedećem je retku riječ od $n$ znakova koji opisuje boju majice svakog
natjecatelja. Svaki znak u toj riječi je jedno od velikih slova \texttt{C}
ili \texttt{Z} koji označava boju majice $i$-tog natjecatelja (crna ili žuta).
Postojat će barem jedan natjecatelj sa crnom i barem jedan natjecatelj sa žutom
majicom.

U sljedećih se $q$ redaka nalaze po dva prirodna broja $x_i$ $(1 \le x_i \le n)$ i
$d_i$ $(1 \le d_i \le 3 \cdot 10^5)$

Maksimalan broj bodova koje neki natjecatelj može osvojti na natjecanju je
$3 \cdot 10^5$.

%%%%%%%%%%%%%%%%%%%%%%%%%%%%%%%%%%%%%%%%%%%%%%%%%%%%%%%%%%%%%%%%%%%%%%
% Output
\subsection*{Izlazni podaci}
U $i$-ti od $q$ redaka izlaza, ispišite ukupan crni ceh nakon $i$-te promjene.

%%%%%%%%%%%%%%%%%%%%%%%%%%%%%%%%%%%%%%%%%%%%%%%%%%%%%%%%%%%%%%%%%%%%%%
% Examples
\subsection*{Probni primjeri}
\begin{tabularx}{\textwidth}{X'X}
  \textbf{ulaz}
  \linespread{1}{\verbatiminput{test/C.dummy.in.1}}
  \textbf{izlaz}
  \linespread{1}{\verbatiminput{test/C.dummy.out.1}} &
  \textbf{ulaz}
  \linespread{1}{\verbatiminput{test/C.dummy.in.2}}
  \textbf{izlaz}
  \linespread{1}{\verbatiminput{test/C.dummy.out.2}}
\end{tabularx}

%%%%%%%%%%%%%%%%%%%%%%%%%%%%%%%%%%%%%%%%%%%%%%%%%%%%%%%%%%%%%%%%%%%%%%
% We're done
\end{statement}

%%% Local Variables:
%%% mode: latex
%%% mode: flyspell
%%% ispell-local-dictionary: "croatian"
%%% TeX-master: "../studentsko2018.tex"
%%% End:

%%%%%%%%%%%%%%%%%%%%%%%%%%%%%%%%%%%%%%%%%%%%%%%%%%%%%%%%%%%%%%%%%%%%%%
% Problem statement

\begin{statement}[
  timelimit=1 s,
  memorylimit=512 MiB,
]{D: Dramatični Dvoboj}

Podne. Livada ispred fotogenične zgrade imenovane po Heydar Aliyevu. Gospodin
Malnar, protagonist ove anegdote, naređuje svojoj vjernoj postrojbi olimpijaca
da se \textbf{preklope}. U tom trenutku jedan hrabar gospodin, vodič hrvatskog
tima i također antagonist, žurnim korakom prilazi gospodinu Malnaru. Nitko nije
znao da je to bio početak epskog sukoba o kojem će se pričati godinama.

Kako bi riješili taj sukob, odlučiše zaigrati igru. Opće je poznato da u
Azerbajdžanu postoji $n$ različitih vrsta pravokutnih tepiha te da $i$-ta vrsta
ima stranice $S$ i $D$, duljina $s_i$ i $d_i$. Igrači na raspolaganju imaju
beskonačno mnogo tepiha svake vrste. Igra započinje tako da gospodin Malnar
na tlo postavi točno $k$ tepiha i to tako da $i$-ti postavljeni tepih bude
tepih vrste $a_i$. Pritom, gospodin Malnar može odrediti hoće li pojedini tepih
postaviti tako da mu je stranica $S$ ili stranica $D$ paralelna s ekvatorom.

U nastavku igre, igrači poteze vuku naizmjence počevši s vodičem, a igru gubi
onaj igrač koji ne može napraviti potez. U jednom potezu, igrač odabire potpuno
novi tepih bilo koje vrste te ga pokušava staviti na jedan od tepiha koji se
nalaze na vrhovima $k$ hrpa tepiha na tlu. Taj se tepih postavlja tako da mu
je stranica $S$ paralelna s ekvatorom te mora \textbf{u potpunosti} ležati na
tepihu na kojeg je postavljen. Odnosno, kada se tepih $P$ stavlja na tepih $Q$,
tada stranica tepiha $P$ koja je paralelna s ekvatorom treba biti strogo manja
od stranice tepiha $Q$ koja je paralelna s ekvatorom, a stranica tepiha $P$
koja je okomita na ekvator treba biti strogo manja od stranice tepiha $Q$ koja
je okomita na ekvator.

Naravno, gospodin Malnar je odigrao savršeno, a i na veliko iznenađenje njegov
protivnik. Nažalost, iz podataka sačuvanih o dvoboju, poznate su brojke $s_i$ i
$d_i$ svih vrsta tepiha te su poznate vrste svih $k$ tepiha koji su bili na tlu.
Jedina je nepoznanica kako je gospodin Malnar orijentirao početnih $k$ tepiha.
Odredite neki od mogućih načina postavljanja prvih $k$ tepiha koji bi mu osigurao
pobjedu.

%%%%%%%%%%%%%%%%%%%%%%%%%%%%%%%%%%%%%%%%%%%%%%%%%%%%%%%%%%%%%%%%%%%%%%
% Input
\subsection*{Ulazni podaci}
U prvom su retku cijeli brojevi $n$ i $k$ $(1 \le n, k \le 10^5)$.

U sljedećih se $n$ redaka nalaze po dva broja $s_i$ i $d_i$
$(1 \le s_i, d_i \le 10^5)$ iz teksta zadatka. Također je poznato
da niti jedna dva tepiha nemaju istu širinu te da niti jedna dva tepiha
nemaju istu dužinu.

U posljednjem se retku nalazi $k$ brojeva $a_i$ $(1 \le a_i \le n)$ iz
teksta zadatka.

%%%%%%%%%%%%%%%%%%%%%%%%%%%%%%%%%%%%%%%%%%%%%%%%%%%%%%%%%%%%%%%%%%%%%%
% Output
\subsection*{Izlazni podaci}
Ako nije moguće postaviti početnih $k$ tepiha tako da gospodin Malnar pobjedi,
u jedini redak ispišite \texttt{"nemoguce"} (bez navodnika).
Inače, u jedini redak ispišite niz nula i jedinica duljine $k$. Ako je $i$-ti broj
jedinica, tada je gospodin Malnar $i$-ti tepih postavio tako da mu je stranica
$D$ paralelna s ekvatorom, a inače ga je postavio tako da mu je stranica $S$
paralelna s ekvatorom.

%%%%%%%%%%%%%%%%%%%%%%%%%%%%%%%%%%%%%%%%%%%%%%%%%%%%%%%%%%%%%%%%%%%%%%
% Examples
\subsection*{Probni primjeri}
\begin{tabularx}{\textwidth}{XX'XX}
  \textbf{ulaz}
  \linespread{1}{\verbatiminput{test/D.dummy.in.1}} &
  \textbf{izlaz}
  \linespread{1}{\verbatiminput{test/D.dummy.out.1}} &
  \textbf{ulaz}
  \linespread{1}{\verbatiminput{test/D.dummy.in.2}} &
  \textbf{izlaz}
  \linespread{1}{\verbatiminput{test/D.dummy.out.2}}
\end{tabularx}

%%%%%%%%%%%%%%%%%%%%%%%%%%%%%%%%%%%%%%%%%%%%%%%%%%%%%%%%%%%%%%%%%%%%%%
% We're done
\end{statement}

%%% Local Variables:
%%% mode: latex
%%% mode: flyspell
%%% ispell-local-dictionary: "croatian"
%%% TeX-master: "../studentsko2018.tex"
%%% End:

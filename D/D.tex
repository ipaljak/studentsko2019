%%%%%%%%%%%%%%%%%%%%%%%%%%%%%%%%%%%%%%%%%%%%%%%%%%%%%%%%%%%%%%%%%%%%%%
% Problem statement

\begin{statement}[
  timelimit=1 s,
  memorylimit=512 MiB,
]{D: Dramatični Dvoboj}

Podne. Livada ispred fotogenične zgrade imenovane po Heydar Aliyevu. Gospodin
Malnar, protagonist ove anegdote, naređuje svojoj vjernoj postrojbi olimpijaca
da se \textbf{preklope}. U tom trenutku jedan hrabar gospodin, vodič hrvatskog
tima i također antagonist, žurnim korakom prilazi gospodinu Malnaru. Nitko nije
znao da je to bio početak epskog sukoba o kojem će se pričati godinama.

Kako bi riješili taj sukob, odlučiše zaigrati igru. Opće je poznato da u
Azerbajdžanu postoji $n$ različitih vrsta pravokutnih tepiha te da $i$-ta vrsta
ima stranice $S$ i $D$, duljina $s_i$ i $d_i$. Iz nekog neobičnog razloga,
tepisima se uvijek stranica $S$ postavlja paralelno s ekvatorom. Za potrebe
igre, postoji $k$ hrpa tepiha na tlu te se na početku igre $i$-ta hrpa sastoji
samo od jednog tepiha vrste $a_i$. U svakom potezu igrač pažljivo odabire tepih
neke vrste te ga postavlja na neku od hrpa tepiha na tlu. Prilikom postavljanja
tepiha je bitno da tepih koji se trenutno nalazi na vrhu ima strogo veću širinu
i dužinu od tepiha koji se stavlja na nj. Igrači poteze vuku naizmjence, a igru
započinje vodič (gospodin Malnar je ipak gospodin).

Večer. Pred sam dvoboj gospodin Malnar je, kontrolirajući vjetar, zarotirao neke
od $k$ tepiha koji se nalaze na tlu. Tim tepisima i samo tim tepisima je
stranica $D$ usporedna s ekvatorom. Za sve ostale tepihe te za sve tepihe kojima
će se rukovati tijekom igre mora vrijediti da im po pravilima prirode stranica
$S$ bude usporedna s ekvatorom.

Naravno, gospodin Malnar je odigrao savršeno, a i na veliko iznenađenje njegov
protivnik, no savršena logika nije dovoljna da biste porazili gospodina
Malnara. Nažalost, iz podataka sačuvanih o dvoboju, poznate su brojke $s_i$ i
$d_i$ svih vrsta tepiha te su poznate vrste svih $k$ tepiha koji su bili na tlu.
Jedina je nepoznanica kako je Malnarov vjetar utjecao na dvoboj. Zato vas molimo
da na temelju tih podataka odredite neki mogući skup tepiha koje je gospodin
Malnar zarotirao svojim poznatim vjetrom kako bi u konačnici pobjedio u igri.

%%%%%%%%%%%%%%%%%%%%%%%%%%%%%%%%%%%%%%%%%%%%%%%%%%%%%%%%%%%%%%%%%%%%%%
% Input
\subsection*{Ulazni podaci}
U prvom su retku cijeli brojevi $n$ i $k$ $(1 \le n, k \le 10^5)$.

U sljedećih se $n$ redaka nalaze po dva broja $s_i$ i $d_i$
$(1 \le s_i, d_i \le 10^5)$ iz teksta zadatka. Također je poznato
da niti jedna dva tepiha nemaju istu širinu te da niti jedna dva tepiha
nemaju istu dužinu.

U posljednjem se retku nalazi $k$ brojeva $a_i$ $(1 \le a_i \le n)$ iz
teksta zadatka.

Ulazni podaci su takvi da rješenje, iako ne nužno jedinstveno, uvijek
postoji.

%%%%%%%%%%%%%%%%%%%%%%%%%%%%%%%%%%%%%%%%%%%%%%%%%%%%%%%%%%%%%%%%%%%%%%
% Output
\subsection*{Izlazni podaci}
U jedini redak ispišite niz nula i jedinica duljine $k$. Ako je $i$-ti broj
jedinica, tada je gospodin Malnar svojim vjetrom zarotirao $i$-ti tepih na
tlu.

%%%%%%%%%%%%%%%%%%%%%%%%%%%%%%%%%%%%%%%%%%%%%%%%%%%%%%%%%%%%%%%%%%%%%%
% Examples
\subsection*{Probni primjeri}
%\begin{tabularx}{\textwidth}{X'X'X}
  %\textbf{ulaz}
  %\linespread{1}{\verbatiminput{test/I.dummy.in.1}}
  %\textbf{izlaz}
  %\linespread{1}{\verbatiminput{test/I.dummy.out.1}} &
  %\textbf{ulaz}
  %\linespread{1}{\verbatiminput{test/I.dummy.in.2}}
  %\textbf{izlaz}
  %\linespread{1}{\verbatiminput{test/I.dummy.out.2}} &
  %\textbf{ulaz}
  %\linespread{1}{\verbatiminput{test/I.dummy.in.3}}
  %\textbf{izlaz}
  %\linespread{1}{\verbatiminput{test/I.dummy.out.3}}
%\end{tabularx}

\textbf{TODO}

%%%%%%%%%%%%%%%%%%%%%%%%%%%%%%%%%%%%%%%%%%%%%%%%%%%%%%%%%%%%%%%%%%%%%%
% We're done
\end{statement}

%%% Local Variables:
%%% mode: latex
%%% mode: flyspell
%%% ispell-local-dictionary: "croatian"
%%% TeX-master: "../studentsko2018.tex"
%%% End:

%%%%%%%%%%%%%%%%%%%%%%%%%%%%%%%%%%%%%%%%%%%%%%%%%%%%%%%%%%%%%%%%%%%%%%
% Problem statement
\begin{statement}[
  timelimit=1 s,
  memorylimit=512 MiB,
]{E: Elokventni Evaluator}

U maloj kućici na livadi, u kojoj jela s roštilja po recepturi starih
leskovačkih majstora priprema Kostadin Stefanović, objeduju Matej i gospodin
Malnar. Ovoga puta je Matej sazvao sastanak, a tema razgovora je izrada novog
sustava za evaluaciju, starog je ipak vrijeme donekle pregazilo.
Gospodin Malnar se sa svime složi te nadoda: ``...i evaluator mora biti
elokventniji.''. ``Elokventniji? Kako to misliš elokventniji?! Jesi li siguran
da su ovo vrganji?'', odgovorio je Matej. Gospodin Malnar mu tada objasni kako mu je
dosta gledanja u jedne te iste poruke poput \texttt{Točno!},
\texttt{Prekoračeno vremensko ograničenje!} ili \texttt{Pogreška pri kompilaciji!}.
Evaluatori bi se trebali prilagoditi zadatku i ponekad prokomentirati izlaz
natjecateljeva programa, ipak je budućnost u umjetnoj inteligenciji. Vaš je
zadatak pomoći Mateju da izradi prototip jednog takvog evaluatora.

Zamislite zadatak u kojem natjecatelj u jednoj liniji mora ispisati matematički
izraz oblika:
\begin{center}
\texttt{<broj><operacija><broj>=<broj>}
\end{center}
pri čemu \texttt{<broj>}
označava bilo koji pozitivan cijeli broj manji ili jednak $10^9$ bez vodećih nula, a
\texttt{<operacija>} je jedan od znakova \texttt{+}, \texttt{-}, \texttt{*} ili
\texttt{/} koji predstavlja jednu od četiri osnovne matematičke operacije. Izraz
koji zadovoljava ova svojstva je \textit{dobro formatiran}.
Elokventni će evaluator na ovom zadatku prikazati jednu od sljedećih poruka:

\begin{itemize}
  \item \texttt{Tocno} -- ispisan je matematički ispravan izraz koji je dobro formatiran.
  \item \texttt{Izraz nije ispravno formatiran} -- ispisani izraz ne odgovara zadanom formatu.
  \item \texttt{Netocno, umjesto <izraz1> mogli ste ispisati <izraz2>} --
    ispisani izraz (\texttt{<izraz1>}) je dobro formatiran, ali nije
    matematički točan te je promjenom \textbf{najviše dva} znaka u ispisu bilo
    moguće dobiti potpuno (matematički i formatom) ispravan izraz (\texttt{<izraz2>}).
  \item \texttt{Potpuno netocno} -- ispisani izraz je dobro formatiran,
    matematički nije točan i nije ga moguće ispraviti koristeći najviše dvije
    zamjene znakova.
\end{itemize}
%%%%%%%%%%%%%%%%%%%%%%%%%%%%%%%%%%%%%%%%%%%%%%%%%%%%%%%%%%%%%%%%%%%%%%
% Input
\subsection*{Ulazni podaci}
U prvoj se liniji nalazi riječ od najviše $30$ znakova koja predstavlja
natjecateljevo rješenje iz teksta zadatka. Ta riječ će se sastojati isključivo
od dekadskih znamenaka (\texttt{0}, \texttt{1}, \dots, \texttt{9}), osnovnih
računskih operatora (\texttt{+}, \texttt{-}, \texttt{*}, \texttt{/}) i znaka
jednakosti (\texttt{=}).

%%%%%%%%%%%%%%%%%%%%%%%%%%%%%%%%%%%%%%%%%%%%%%%%%%%%%%%%%%%%%%%%%%%%%%
% Output
\subsection*{Izlazni podaci}
U jedini redak ispišite odgovarajuću poruku elokventnog evaluatora iz
teksta zadatka.

%%%%%%%%%%%%%%%%%%%%%%%%%%%%%%%%%%%%%%%%%%%%%%%%%%%%%%%%%%%%%%%%%%%%%%
% Examples
\subsection*{Probni primjeri}
  \begin{tabularx}{\textwidth}{X'X}
  \textbf{ulaz}
  \linespread{1}{\verbatiminput{test/E.dummy.in.1}}
  \textbf{izlaz}
  \linespread{1}{\verbatiminput{test/E.dummy.out.1}} &
  \textbf{ulaz}
  \linespread{1}{\verbatiminput{test/E.dummy.in.2}}
  \textbf{izlaz}
  \linespread{1}{\verbatiminput{test/E.dummy.out.2}}
\end{tabularx}

%%%%%%%%%%%%%%%%%%%%%%%%%%%%%%%%%%%%%%%%%%%%%%%%%%%%%%%%%%%%%%%%%%%%%%
% We're done
\end{statement}

%%% Local Variables:
%%% mode: latex
%%% mode: flyspell
%%% ispell-local-dictionary: "croatian"
%%% TeX-master: "../studentsko2018.tex"
%%% End:

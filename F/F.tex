%%%%%%%%%%%%%%%%%%%%%%%%%%%%%%%%%%%%%%%%%%%%%%%%%%%%%%%%%%%%%%%%%%%%%%
% Problem statement
\begin{statement}[
  timelimit=1 s,
  memorylimit=512 MiB,
]{F: Fantastični Fožgaj}

Fožgaj se želi sprijateljiti s Dodom i u tu svrhu je odlučio sastaviti govor o
prijateljstvu.

Fožgajev govor je riječ koja se sastoji od $m$ malih slova engleske abecede.
Domagoj (koji je već od Fožgaja čuo jedan govor o prijateljstvu) zna da postoji
popis od $n$ riječi koje bi mogle preplašiti malog Dodu.

Gospodin Malnar je uspio Fožgaju nabaviti taj popis, a sada Fožgaja zanima
koliko postoji različitih riječi duljine $m$ u kojima se ne pojavljuje nijedna
riječ s popisa. Za riječ s popisa kažemo da se pojavljuje u Fožgajevom govoru
ako je u njemu prisutna kao uzastopan niz znakova.

Pomozite Fožgaju i ispišite koliko postoji različitih riječi duljine $m$ u kojoj
se ne pojavljuje nijedna riječ s popisa, modulo $10^9 + 7$.


%%%%%%%%%%%%%%%%%%%%%%%%%%%%%%%%%%%%%%%%%%%%%%%%%%%%%%%%%%%%%%%%%%%%%%
% Input
\subsection*{Ulazni podaci}
U prvom se retku nalaze prirodni brojevi $n$ $(1 \le n \le 100)$ i
$m$ $(1 \le m \le 10^9)$ iz teksta zadatka.

U svakom od sljedećih $n$ redaka nalazi se jedna riječ s Domagojevog popisa.
Riječi s popisa sastoje se od malih slova engleske abecede i neće se ponavljati.
Također, ukupni broj znakova na popisu manji je ili jednak $100$.

%%%%%%%%%%%%%%%%%%%%%%%%%%%%%%%%%%%%%%%%%%%%%%%%%%%%%%%%%%%%%%%%%%%%%%
% Output
\subsection*{Izlazni podaci}
U jedini redak ispišite traženi broj iz teksta zadatka.

%%%%%%%%%%%%%%%%%%%%%%%%%%%%%%%%%%%%%%%%%%%%%%%%%%%%%%%%%%%%%%%%%%%%%%
% Examples
\subsection*{Probni primjeri}
\begin{tabularx}{\textwidth}{X'X'X}
  \textbf{ulaz}
  \linespread{1}{\verbatiminput{test/F.dummy.in.1}}
  \textbf{izlaz}
  \linespread{1}{\verbatiminput{test/F.dummy.out.1}} &
  \textbf{ulaz}
  \linespread{1}{\verbatiminput{test/F.dummy.in.2}}
  \textbf{izlaz}
  \linespread{1}{\verbatiminput{test/F.dummy.out.2}} &
  \textbf{ulaz}
  \linespread{1}{\verbatiminput{test/F.dummy.in.3}}
  \textbf{izlaz}
  \linespread{1}{\verbatiminput{test/F.dummy.out.3}}
\end{tabularx}

%%%%%%%%%%%%%%%%%%%%%%%%%%%%%%%%%%%%%%%%%%%%%%%%%%%%%%%%%%%%%%%%%%%%%%
% We're done
\end{statement}

%%% Local Variables:
%%% mode: latex
%%% mode: flyspell
%%% ispell-local-dictionary: "croatian"
%%% TeX-master: "../studentsko2018.tex"
%%% End:

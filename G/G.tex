%%%%%%%%%%%%%%%%%%%%%%%%%%%%%%%%%%%%%%%%%%%%%%%%%%%%%%%%%%%%%%%%%%%%%%
% Problem statement
\begin{statement}[
  timelimit=1 s,
  memorylimit=512 MiB,
]{G: Golema Gozba}

Na gozbu je pozvano $2n$ studenata. Oni sjede oko okruglog stola i označeni su
redom u smjeru kazaljke na satu od $1$ do $2n$. Student s oznakom $j$ sjedi
pored studenata s oznakama $(j - 1)$ te $(j + 1)$, gdje smatramo da su studenti
s oznakama $1$ i $2n$ također susjedni. Dodatno, studenti su podijeljeni u $n$
parova najboljih prijatelja. Parovi su disjunktni, dakle svaki student je u
točno jednom paru.

Gospodin Malnar će odrediti koje je jelo najbolje za svakog studenta. Danas su
u ponudi dvije vrste jela, $A$ i $B$, a gospodin Malnar zbog raznolikosti jela
želi napraviti odabir koji zadovoljava sljedeće:
\begin{itemize}
    \item Svaki student će jesti točno jedno od jela $A$ ili $B$.
    \item Studenti iz istog para moraju jesti različita jela.
    \item Ne smiju postojati tri uzastopna studenta u krugu koja imaju isto jelo.
\end{itemize}

Gospodin Malnar se već dosjetio jednog rasporeda, a sada ga zanima možete li i
vi pronaći jedan. Ispišite bilo koji odabir jela za studente koji zadovoljava
navedene uvjete, a u slučaju da se gospodin Malnar našalio i takav raspored ne
postoji, ispišite \texttt{"Malnar se nasalio"} (bez navodnika).

%%%%%%%%%%%%%%%%%%%%%%%%%%%%%%%%%%%%%%%%%%%%%%%%%%%%%%%%%%%%%%%%%%%%%%
% Input
\subsection*{Ulazni podaci}
U prvom je retku prirodan broj $n$ $(2 \le n \le 5 \cdot 10^5)$ - broj parova
studenata.

U sljedećih $n$ redaka nalazi se par različitih prirodnih brojeva $x$ i $y$
$(1 \le x$, $y \le 2n)$ koji označava par studenata koji su najbolji prijatelji.

Svaki student će biti član točno jednog para.
%%%%%%%%%%%%%%%%%%%%%%%%%%%%%%%%%%%%%%%%%%%%%%%%%%%%%%%%%%%%%%%%%%%%%%
% Output
\subsection*{Izlazni podaci}
U slučaju da ne postoji rješenje ispišite \texttt{"Malnar se nasalio"}
(bez navodnika).

U protivnom, u jedini redak ispišite niz od $2n$ znakova \texttt{A} ili
\texttt{B} koji opisuje valjani izbor jela za studente. Po redu $i$-ti znak
treba predstavljati odabir za $i$-tog studenta.

Ako postoji više valjanih odabira, ispišite bilo koji.

%%%%%%%%%%%%%%%%%%%%%%%%%%%%%%%%%%%%%%%%%%%%%%%%%%%%%%%%%%%%%%%%%%%%%%
% Examples
\subsection*{Probni primjeri}
\begin{tabularx}{\textwidth}{X}
  \textbf{ulaz}
  \linespread{1}{\verbatiminput{test/G.dummy.in.1}}
  \textbf{izlaz}
  \linespread{1}{\verbatiminput{test/G.dummy.out.1}}
\end{tabularx}

%%%%%%%%%%%%%%%%%%%%%%%%%%%%%%%%%%%%%%%%%%%%%%%%%%%%%%%%%%%%%%%%%%%%%%
% We're done
\end{statement}

%%% Local Variables:
%%% mode: latex
%%% mode: flyspell
%%% ispell-local-dictionary: "croatian"
%%% TeX-master: "../studentsko2018.tex"
%%% End:

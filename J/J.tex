%%%%%%%%%%%%%%%%%%%%%%%%%%%%%%%%%%%%%%%%%%%%%%%%%%%%%%%%%%%%%%%%%%%%%%
% Problem statement
\begin{statement}[
  timelimit=1 s,
  memorylimit=512 MiB,
]{J: Jaki Jovsi}

Jovsi je jak dječak. Od malena je volio strojnice pa je ih je često volio
imitarti, samo iz nekog razloga nije vikao \textit{trtrtrt} ili
\textit{bambambam}, nego \textit{acacacacac}

Gospodin Malnar želi istovremeno testirati Jovsijevu snagu i sposobnost
rješavanja zadataka. Tako mu je jednog dana poklonio štap na kojemu je od
lijevog do desnog kraja ispisano $n$ slova, označenih redom od prvog do $n$-tog.
Malnar smatra da su simetrični štapovi jako lijepi, zato ga posebno zanimaju
palindromski parovi -- to su uređeni parovi prirodnih brojeva $(l, r)$, gdje
$1 \le l \le r \le n$, takvi da je riječ dobivena gledajući samo slova od $l$-te
do $r$-te pozicije palindrom. Podsjetimo se da je palindrom riječ koja se čita
jednako slijeva nadesno kao i zdesna nalijevo.

Malnar je zatim odlučio Jovsiju zadati izazov. Izazov se sastoji od prirodnog
broja $k$ te niza od $k$ palindromskih parova $(l_i, r_i)$ za koje vrijedi
$l_1 < l_2 < \dots < l_k$ te $r_1 > r_2 > \dots > r_k$. Jovsi mora svojim
snažnim udarcima redom $k$ puta lomiti štap kako bi po Malnarovoj želji nakon
$i$-tog loma od štapa ostao palindromski par $(l_i, r_i)$.

Jovsi mora biti spreman na svaku situaciju pa ga zanima koliko postoji
različitih izazova koje može dobiti od gospodina Malnara. Pomozite Jovsiju i
ispišite koliko postoji različitih izazova, modulo $998244353$.

%%%%%%%%%%%%%%%%%%%%%%%%%%%%%%%%%%%%%%%%%%%%%%%%%%%%%%%%%%%%%%%%%%%%%%
% Input
\subsection*{Ulazni podaci}
U jedinom je retku riječ koja se sastoji od malih slova engleske abecede, a
predstavlja niz slova ispisanih na Malnarovom štapu. Riječ će se sastojati
od najviše milijun znakova.

%%%%%%%%%%%%%%%%%%%%%%%%%%%%%%%%%%%%%%%%%%%%%%%%%%%%%%%%%%%%%%%%%%%%%%
% Output
\subsection*{Izlazni podaci}
U jedinom retku potrebno je ispisati ostatak pri dijeljenju broja različith
izazova s $998244353$.

%%%%%%%%%%%%%%%%%%%%%%%%%%%%%%%%%%%%%%%%%%%%%%%%%%%%%%%%%%%%%%%%%%%%%%
% Examples
\subsection*{Probni primjeri}
\begin{tabularx}{\textwidth}{X'X'X}
  \textbf{ulaz}
  \linespread{1}{\verbatiminput{test/J.dummy.in.1}}
  \textbf{izlaz}
  \linespread{1}{\verbatiminput{test/J.dummy.out.1}} &
  \textbf{ulaz}
  \linespread{1}{\verbatiminput{test/J.dummy.in.2}}
  \textbf{izlaz}
  \linespread{1}{\verbatiminput{test/J.dummy.out.2}} &
  \textbf{ulaz}
  \linespread{1}{\verbatiminput{test/J.dummy.in.3}}
  \textbf{izlaz}
  \linespread{1}{\verbatiminput{test/J.dummy.out.3}}
\end{tabularx}

%%%%%%%%%%%%%%%%%%%%%%%%%%%%%%%%%%%%%%%%%%%%%%%%%%%%%%%%%%%%%%%%%%%%%%
% We're done
\end{statement}

%%% Local Variables:
%%% mode: latex
%%% mode: flyspell
%%% ispell-local-dictionary: "croatian"
%%% TeX-master: "../studentsko2018.tex"
%%% End:

\documentclass[a4paper]{article}
\usepackage{zadaci}
\usepackage[dvipsnames]{xcolor}
\usepackage{wrapfig}
\usepackage{url}
\usepackage{tikz}
\usepackage{amsmath}
\usepackage[normalem]{ulem}
\usetikzlibrary{angles,quotes}
\contestname{Studentsko natjecanje\\27. listopada 2019.}
\markright{\textbf{\textsf{Opisi algoritama}}}

\begin{document}

\section*{Opisi algoritama}
Zadatke, testne primjere i rješenja pripremili: Patrick Pavić, Ivan Paljak i
Krešimir Nežmah. Prirmjeri implementiranih rješenja dani su u priloženim izvornim
kodovima.

\subsection*{Zadatak A -- Alergični Aron}
\textsf{Predložio: Patrick Pavić}\\
\textsf{Potrebno znanje: teorija grafova, union-find}

Probat ćemo za svaki brid naći najveću komponentu u kojoj je upravo on brid s
najmanjom težinom.

Soritrajmo sve bridove po težinama. Dodavat ćemo ih od najvećih prema manjima u
određenu strukturu na način da možemo saznati veličinu komponente u kojoj se
trenutno nalazi čvor te imati operaciju dodavanja brida. Te operacije podržava
struktura union-find.

Primijetimo da nakon spajanja s novim bridom $e$, svi bridovi u komponenti imaju
vrijednost veću ili jednaku vrijednosti tog brida. Dakle, nakon spajanja možemo
samo pogledati veličinu komponente i pomnožiti s težinom tog brida.

Uzmemo li maksimum tog umnoška po svim bridovima, dobili smo rješenje. Time smo
dobili složenost $\mathcal{O}(n\log{}n)$ zbog sortiranja bridova.

\subsection*{Zadatak B -- Bliski Brojevi}
\textsf{Predložio: Patrick Pavić}\\
\textsf{Potrebno znanje: Moov algoritam, turnirsko stablo}

Primjenimo Moov algoritam na upite. Sada trebamo podržati operacije:
dodaj broj u strukturu, izbaci broj iz strukture, nađi par s najmanjom
apsolutnom razlikom.

To možemo izvesti pomoću turnirskog stabla.

Turnirsko stablo će u čvoru koji prikazuje interval $[l, r]$ održavati najmanji
broj $l'$ iz tog intervala, najveći broj $r'$ iz toga intervala te najmanju
apsolutnu razliku među dva broja koja se oba nalaze u tom intervalu. Primjetimo
da pri spajanju dva djeteta u njihovog roditelja u turnirskom stablu trivijalno
možemo izračunati nove $l'$ i $r'$, a najmanji odgovor će biti ili jedan od dva
odgovora djece, ili razlika među $r'$ lijevog djeteta i $l'$ desnog djeteta.
Time imamo strukturu koja u $\mathcal{O}(\log{}n)$ podržava ubacivanje,
izbacivanje i upit.

U Moovom algoritmu napravit ćemo $\mathcal{O}(n\sqrt{n})$ operacija nad
strukturom što daje složenost $\mathcal{O}(n\sqrt{n}\cdot \log{} n)$. Rješenje
je bilo potrebno implementirati s malom konstantom kako bi prošlo ograničenja.

Postoji i $\mathcal{O}(n\sqrt{n})$ rješenje koje se ostavlja čitatelju za vježbu.

Pomoć: podijelite upite na veće i manje od $\sqrt{n}$.
\clearpage

\subsection*{Zadatak C -- Crni Ceh}
\textsf{Predložio: Krešimir Nežmah}\\
\textsf{Potrebno znanje: Fenwickovo ili turnirsko stablo}

Na početku je ukupan crni ceh jednak $0$. Izračunat ćemo koliko se promijenio
odgovor nakon svake promjene u rezultatima.

Ako natjecatelj u crnoj majici dobije bodove, potrebno je pronaći od koliko je
natjecatelja u žutim majicama on upravo postao bolji te za toliko povećati
odgovor. Sličan postupak radimo ako natjecatelj u žutoj majici dobije bodove.

Preostaje brzo izračunati koliko postoji natjecatelja čiji je broj bodova manji
ili jednak nekoj vrijednosti. To možemo koristeći \textit{Fenwickovo stablo},
po jedno za svaku boju majice. Budući da je broj bodova natjecatelja uvijek
između $0$ i $3\cdot10^5$, nema problema s održavanjem te strukture u memoriji.

Vremenska složenost je $\mathcal{O}(n + q \log{} \text{MAXD})$.

\subsection*{Zadatak D -- Dramatični Dvoboj}
\textsf{Predložio: Patrick Pavić}\\
\textsf{Potrebno znanje: Grundyev teorem, Fenwickovo stablo, Gaussova eliminacija}

Definirajmo stanje igre kao tepih na vrhu. Izračunajmo Grundyeve brojeve za sve
tepihe te njihove zarotirane inačice. Primijetimo da, ako tepih $i$ možemo
staviti na tepih $j$ te tepih $j$ na tepih $k$, onda možemo staviti tepih $i$
na tepih $k$. Odnosno, uvjet mogućnosti "stavljanja" je tranzitivan.
Stoga, umjesto uzimanja \textit{mex}-a svih stanja u koje možemo otići,
jednostavno možemo uzeti maksimum te ga uvećati za jedan.

Grundyeve brojeve možemo izračunati tako da sve tepihe sortiramo uzlazno po $S_i$
te održavamo Fenwickovo stablo u koje ubacujemo njihove Grundy brojeve na
poziciju $D_i$.  Na taj način kada dođemo do nekog tepiha, pogledano maksimum
pomoću Fenwickovog stabla u intervalu $[0, D_i>$ te ga uvećamo za jedan i time
dobijemo Grundy broj tog tepiha. Ako taj tepih nije zarotirana inačica, onda je
on moguće stanje u koje možemo doći iz nekog drugog stanja te ga dodajemo u
Fenwickovo stablo.

Neka je za $i$-ti tepih na tlu, $G_i$ njegov Grundy broj, a $G'_i$ Grundy broj
njega zarotiranog. Po Grundy teoremu, stanje cijele igre je xor svih
pojedinačnih stanja, a drugi igrač pobjeđuje ako je broj nula.
Stoga, želimo odabrati brojeve na način da je taj xor nula. Prvo odabiremo
svugdje broj $G_i$ te xoranjem tih brojeva dobijemo broj $X$. Sada, ako pomoću
brojeva ($G_i$ xor $G'_i$) možemo dobiti podskup čiji je xor $X$, onda na tim
mjestima koja su u skupu zarotiramo tepih i time smo dobili rješenje. To možemo
riješiti Gaussovom eliminacijom, ako operaciju xora promatramo kao zbrajanje
vektora modulo 2.

\subsection*{Zadatak E -- Elokventni Evaluator}
\textsf{Predložio: Ivan Paljak}\\
\textsf{Potrebno znanje: brute-force}

Ovo je u potpunosti implementacijski zadatak. Da biste na našem,
manje elokventnom evaluatoru ugledali poruku {\color{ForestGreen} \textbf{Accepted}},
bilo je potrebno pažljivo pročitati tekst zadatka, implementirati ono što piše u
tekstu zadatka i pokriti sve slučajeve. Standardnu (natjecateljsku)
implementaciju možete vidjeti u izvornom kodu \texttt{E.cpp}, a nešto
kompaktniju implementaciju koja koristi regularne izraze i funkciju \texttt{eval}
u Pythonu, možete vidjeti u izvornom kodu \texttt{E.py}.

\clearpage

\subsection*{Zadatak F -- Fantastični Fožgaj}
\textsf{Predložio: Krešimir Nežmah}\\
\textsf{Potrebno znanje: dinamičko programiranje, trie, potenciranje matrica}

Pogledajmo prvo naivno rješenje. Koristimo dinamičko programiranje.
Neka je $dp[s]$ broj mogućih govora ako smo dosad već izgovorili string $s$.
Ako $s$ sadrži neku od riječi s popisa onda je $dp[s] = 0$, a inače, ako je
duljina od $s$ jednaka $m$, tada je $dp[s] = 1$. U prijelazu pokušavamo
produljiti $s$ sa svakim slovom.

Uočimo da je jedini problem ako u nekom trenutku pri prijelazu napravimo riječ
čiji je sufiks neka riječ s popisa. U stanju je, dakle, dovoljno pamtiti duljinu
dosad izgrađene riječi te najdulji sufiks od $s$ koji je neki prefiks riječi s
popisa. Konkretno, ako ubacimo sve riječi s popisa u \textit{trie} (stoblo),
stanje postaje trenutna duljina te pozicija u trieu.

Da bi prijelaz bio brz, potrebno je za svaki čvor u trieu znati dvije stvari:
\begin{itemize}
    \item sadrži li on kao podstring neku riječ s popisa.
    \item kada bismo probali dodati slovo na kraj tog čvora, koji je najdulji
          sufiks tako dobivene riječi koji se nalazi u trieu.
\end{itemize}

Dakle, potrebno je označiti čvorove u trieu koje u dinamici ne smijemo posjetiti
te dodati dodatne bridove prema stanjima koje možemo posjetiti. Takav graf
moguće je izgraditi u linearnoj složenosti, no kako je ukupni broj znakova u
trieu najviše $100$, bilo koja naivna konstrukcija takvog grafa trebala bi stati
unutar vremenskog ograničenja.

Vremenska složenost dinamike trenutno je $\mathcal{O}(nm)$, što je zasad očito
previše. Da bismo je ubrzali, prijelaze možemo bilježiti u $100\times100$ matrici.
Potenciranjem te matrice na $m$-tu potenciju možemo dobiti odgovor za govor
duljine $m$. Koristeći brzo potenciranje matrica, dobivamo vremensku složenost
$\mathcal{O}(n^3 \log{} m)$.

\subsection*{Zadatak G -- Golema Gozba}
\textsf{Predložio: Krešimir Nežmah}\\
\textsf{Potrebno znanje: teorija grafova, konstruktivni algoritmi, dfs}

Tvrdimo da rješenje uvijek postoji (gospodin Malnar ipak uspijeva bez problema
naći raspored). Dokaz je konstruktivan.

Napravimo graf u kojemu su osobe od $1$ do $2n$ čvorovi. Ako su dvije osobe u
istom paru, povezat ćemo ih crvenim bridom. Dodatno, povezat ćemo plavim bridom
osobe u parovima $(1, 2)$, $(3, 4)$, \dots, $(2n-1, 2n)$. Uočimo da ne postoje
dva brida iste boje koji dijele krajnju točku. Zato će svaki ciklus u ovakvom
grafu imati alternirajuće boje (crvena, plava, crvena, plava, \dots), pa je
svaki ciklus parne duljine. Iz toga slijedi da je graf bipartitan te bojenje
možemo pronaći jednostavnim dfs algoritmom.

Uvjeti zadatka sada su očito zadovoljeni -- crveni bridovi osiguravaju da dvije
osobe iz istog para nemaju isto jelo, a plavi bridovi osiguravaju da među tri
uzastopne pozicije u krugu nema tri ista jela.

Vremenska složenost je $\mathcal{O}(n)$.

\clearpage

\subsection*{Zadatak H -- Herojski Histogram}
\textsf{Predložio: Krešimir Nežmah}\\
\textsf{Potrebno znanje: convex hull trick}

Izračunat ćemo za svaki stupac $j$, koja je najveća površina pravokutnika čija
je desna granica upravo stupac $j$. Nakon što smo to napravili, odgovor za prvih
$j$ stupaca je maksimum među tim vrijednostima od $1$ do $j$.

  Za svaki stupac pronađimo prvi stupac lijevo od njega koji je manji od njega, za
stupac $j$ neka je to $left[j]$, gdje uzimamo $left[j] = 0$ ako takav stupac ne
postoji. Promatrajmo neki stupac na indeksu $p_1$ s visinom $h_1$. Možemo
pratiti padajući niz stupaca $p_2 = left[p_1]$, $p_3 = left[p_2]$, \dots,
$p_m = left[p_{m-1}]$ dok ne dođemo do kraja ($p_m = 0$). Njihove visine neka su
redom $h_1$, $h_2$, \dots, $h_m$. Svakom pravokutniku čija je desna granica
$p_1$, možemo pomicati lijevu granicu dok ne dođemo do nekog $p_i$, čime samo
povećavamo površinu. Dakle, dovoljno je razmatrati samo one pravokutnike čija je
lijeva granica za jedan desno nekog od indeksa $p_1$, $p_2$, \dots, $p_m$.
Odgovor za stupac $p_1$ je maksimum od vrijednosti $(p_1 - p_{i+1}) \cdot h_i$, za
$i$ između $1$ i $m-1$, što možemo napisati i kao $h_i \cdot p_1 - h_i \cdot p_{i+1}$.

Definirajmo zato za svaki stupac funkciju -- za stupac $i$ imamo
$f_i(x) = h_i \cdot x - h_i \cdot left[i]$. Tada vidimo da je odgovor za stupac $p_1$
zapravo maksimum od vrijednosti $f_{p_1}(p_1), f_{p_2}(p_1), \dots, f_{p_{m-1}} (p_1)$.
Budući da su sve navedene funkcije linearne, ovo je poznati problem u kojem je
potrebno održavati gornji envelope danih pravaca (tzv. \textit{convex hull trick}).

No problem je što za različite početne $p_1$, imamo različite ljuske i ne
možemo izgraditi svaku ispočetka. Konkretno, ako promatramo vezu $j$ i $left[j]$
kao dijete i roditelj, dobivamo stablo ukorijenjeno s pozicijom $0$, a da bismo
dobili odgovor za neku poziciju, potrebno je promatrati konveksnu ljusku dobivenu
prateći put od te pozicije do korijena. Uočimo dodatno da na putu od korijena do
bilo kojeg čvora visine stupaca rastu, dakle nagibi pravaca su rastući.

Pokrenut ćemo dfs obilazak stabla i održavati trenutnu ljusku pravaca. Moramo
podržati dvije operacije:

\begin{itemize}
  \item prelazak s roditelja na dijete -- u ljusku dodajemo novi pravac, a prije
        toga smo ih nekoliko morali izbrisati (uvijek brišemo sufiks jer na putu
        od korijena su nagibi pravaca rastući).
  \item prelazak s djeteta na roditelja -- moramo vratiti ljusku u originalno stanje.
\end{itemize}

Naivno brisanje pravaca daje kvadratnu složenost u slučaju da u stablu postoji
dugački lanac koji završava s čvorom koji ima veliki stupanj.

Da bismo efikasno napravili promjene, pri ubacivanju možemo binarnim
pretraživanjem pronaći poziciju u ljusci gdje je potrebno ubaciti novi pravac.
Potrebno je zabilježiti tu promjenu te ažurirati veličinu ljuske, a pri povratku
je onda lagano rekonstruirati originalnu ljusku.

Za više detalja pogledajte i zadatak \textit{Harbingers} sa CEOI 2009 koji je
popularizirao ovaj algoritam u svijetu natjecateljskog programiranja.

Vremenska složenost je $\mathcal{O}(n \log{}n)$.

\clearpage

\subsection*{Zadatak I -- Idilični Instagram}
\textsf{Predložio: Ivan Paljak}\\
\textsf{Potrebno znanje: brute-force}

Zamislimo da su sve slike s Alenkinog instagram profila poredane u niz duljine
$n$ redoslijedom objava. Lagano je uočiti da su slike dobro preklopljene ako za
svako putovanje osim kronološki najstarijeg (sufiks niza) vrijedi da je broj
objavljenih slika s tog putovanja višekratnik broja tri.

Budući da broj različitih putovanja ne može biti veći od $26$ (broj slova u
engleskoj abecedi), možemo za svako putovanje pretpostaviti da će upravo ono
biti najstarije nakon brisanja nekih slika. Broj slika na novom profilu u tom je
slučaju jedinstveno određen. Sva starija putovanja od fiksiranog najstarijeg
moraju biti u potpunosti obrisana, a za svako novije putovanje je potrebno
obrisati najmanji broj slika tako da broj slika koje ostanu bude višekrantik
broja tri. Jasno, taj broj odgovara ostatku pri dijeljenju broja slika sa tri.

Ovaj postupak je konstruktivan pa je i rješenje zadatka potpuno. Ovaj algoritam
moguće je implementirati u vremenskoj složenosti $\mathcal{O}(n)$.

\subsection*{Zadatak J -- Jaki Jovsi}
\textsf{Predložio: Patrick Pavić}\\
\textsf{Potrebno znanje: olbat stablo, dinamičko programiranje}

Izgradimo \textit{olbat stablo} (\textit{palindromsko stablo}, \textit{eertree})
nad nizom znakova. Sada ćemo dinamičkim programiranjen izračunati odgovor za
svaki čvor (tj.\ palindrom) te ga pomnožiti s brojem pojavljivanja u nizu.
Za svaki čvor $X$ postoji čvor $BK$ -- najveći palindrom koji je sufiks
palindroma $X$ te $PAR$ što je palindrom koji nastaje micanjem prvog i zadnjeg
znaka iz tog palindroma. U olbat stablu to su čvor u koji pokazuje backedge i
čvor koji je roditelj tog čvora, zato oznake $BK$ i $PAR$. Stanje dinamike će 
se sastojati od oznake između $0$ i $2$ te čvora $X$.

Oznaka $1$: rješenje kad bismo u prvom potezu ostavili baš taj palindrom\\
Oznaka $0$: kada bismo ostavili ili taj palindrom ili bilo koji palindrom koji je prefiks početnog.\\
Oznaka $2$: kada bismo u prvom potezu ostavili ili taj palindorm ili bilo koji podpalindorm.\\

Podpalindrom palindroma $X$ je svaki palindrom koji se nalazi kao podriječ unutar $X$.

Prijelazi dinamike su:
\begin{align*}
  dp(X, 1) &= dp(PAR, 2) + 1 \\
  dp(X, 0) &= dp(BK, 0) + dp(X, 1) \\
  dp(X, 2) &= dp(PAR, 2) + dp(X, 1) + 2 \cdot dp(BK, 0)
\end{align*}

Završni odgovor za čvor $X$ spremljen je u $dp(X,1)$.

Vremenska složenost je $\mathcal{O}(n)$.


\end{document}
%%% Local Variables:
%%% mode: latex
%%% mode: flyspell
%%% ispell-local-dictionary: "croatian"
%%% End:
